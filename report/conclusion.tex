\section{Conclusion}
In this work on the Horse colic data set, association rule generation is highly dependent on the available attributes.
Many of the attributes are strongly correlated by definition, which results in trivial rules (such as in section \ref{sec: leve}). After removing such attributes, the Apriori algorithm is a robust method of rule generation.\\

\noindent
With the following parameters
\begin{itemize}
\item Setting $minsup = \left[0.1, 0.42 \right]$
\item Choosing lift as a metric, with $minconf \sim 4.0$
\item Setting the significance level to $0.01$
\end{itemize}
rulesets (with an upper bound of 1000 rules) are generated within 0.5 seconds for all cases. A selection of  interesting rules corresponding to these parameters can be found in section \ref{sec: rule gen}.\\

\noindent
Important rules can be missed when removing attributes such as \verb|age| or \verb|types of lesions|. Reintroducing these attributes for future work may be useful, once more data has been collected for the underrepresented values (e.g. collecting more cases for young horses).\\

\noindent
In order to find the rarest possible rules, minimum support can be lowered to $minsup = \frac{1}{368} \approx 0.003$. However, computational resources on the general-consumer level may not be adequate for this task.\\


\noindent
Additionally, classification with association rule analysis tends to ignore rules where the consequent has low support (e.g. \verb|abdomen| is biased towards \verb|distended lg colon| and a majority of rules use this class value). Other attributes may be more suitable as classes in this case (e.g. \verb|outcome|,\verb|surgery| or \verb|surgical lesion|).\\


\noindent
In general, association rule analysis of this data set has the potential to aid veterinarians in diagnosis in horses who suffer from abdominal pain, as well as offering suggestions when making a prognosis. With enough data, some new insight can be gained from non-invasive measurements and can help identify criteria for costly and dangerous treatments such as surgery.

