\section{Metric-based rule selection}

After removing uneven attributes and discretizing all of the numerical attributes in our data set, we can investigate the rules generated by the Apriori algorithm. 

We have 4 potential metrics for ranking our rules.
\begin{itemize}
\item Confidence
\item Lift
\item Leverage
\item Conviction
\end{itemize}
%
\subsection{Confidence}
Setting a threshold for this metric ($minconf = 0.95$) results in hundreds of high confidence rules.
E.g. running 
\begin{verbatim}
	   weka.associations.Apriori -I -R -N 1000 -T 0 -C 0.95 -D 0.05 -U 1.0 -M 0.1 -S -1.0 -c 24
\end{verbatim}

\begin{itemize}

\item \verb|95. surgery=Treated without surgery |

\verb|peripheral pulse=Normal| 

\verb|site of lesion=0| 

\verb|outcome=Lived 39| 

\verb|   ==>|

\verb|surgical lesion=No 39|

\verb|<conf:(1)> lift:(2.71) lev:(0.07) [24] conv:(24.59)|

%%%%%
%%%%%

\item \verb|105. peristalsis=Absent,|

\verb|abdominal distension=Severe,|

\verb|nasogastric reflux=>1liter,|

\verb|rectal examination - feces=Decreased 39|

\verb|   ==>|

\verb|abdomen=Distended lg intestine 39|

\verb|<conf:(1)> lift:(1.54) lev:(0.04) [13] conv:(13.67)|

%%%%%
%%%%%

\item \verb|844. capillary refill time=<3sec,|

\verb|abdominocentesis appearance=Cloudy,|

\verb|site of lesion=0 40| 

\verb|   ==>|

\verb|surgical lesion=No ,|

\verb|outcome=Lived 38|

\verb|<conf:(0.95)> lift:(3.21) lev:(0.07) [26] conv:(9.38)|
\end{itemize}
%
The first rule can be considered as an obvious rule (non-existing lesions don't require surgical intervention).
However, the next rule has identical support and confidence and can be useful in a diagnostic sense.
The last rule can potentially help with making a prognosis (in select few cases).